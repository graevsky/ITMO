
\usetikzlibrary{circuits}

\setcounter{page}{77}

\graphicspath{ {./images/} }
\usetikzlibrary{circuits}


%\begin{document}



\setlength{\columnsep}{20pt}
\begin{multicols*}{2}

\bgroup
\fontsize{9}{11}\selectfont




\noindent= $\pm \frac{3}{5}$, т. е. либо $\sin^2{x} = \frac{1}{5}$, либо $\sin^2{x} = \frac{4}{5}$.
То и\vspace{0.4em} другое противоречит равенству $\sin{x} = \frac{5\sqrt{2}}{8}$.\vspace{1.7em}\\
\noindentВариант 8.
\\\textbf{1}. $\pi n/3$, $n \in Z$. \textbf{2}. $\pm6$. \textbf{3}. $[1;5)\cup(10;\infty)$.
\textbf{4}. 1:6. У\hspace{0.04cm}к\hspace{0.04cm}а\hspace{0.04cm}з\hspace{0.04cm}а\hspace{0.04cm}н\hspace{0.04cm}и\hspace{0.04cm}е. Пусть S    ---    площадь    параллелограмма \hspace{0.4cm}    $ABCD$.\hspace{0.3cm}      Тогда \hspace{0.3cm} $S_{KBL} = \frac{1}{3}S_{ABL}=\frac{1}{24}S$,\vspace{0.3em} а $S_{BLM} = \frac{1}{4}S$.\vspace{0.3em}
\\\noindent\textbf{5}. $5<a<7$. У\hspace{0.04cm}к\hspace{0.04cm}а\hspace{0.04cm}з\hspace{0.04cm}а\hspace{0.04cm}н\hspace{0.04cm}и\hspace{0.04cm}е.Квадратное уравнение $ax^2+bx+c=0$ имеет два положительных корня тогда и только тогда, когда\vspace{1em}\\
$\begin{cases}
    ac>0,\\
    b^2-4ac>0,\\
    ab<0.
\end{cases}$\vspace{1.7em}\\
\noindent Вариант 9
\\\textbf{1}. Имеют.
\\\textbf{2}. $(-\infty; -5/4]\cup(-1/4$).
\\\textbf{3}. \hspace{0.2cm} $6/23$. \hspace{0.2cm} У\hspace{0.04cm}к\hspace{0.04cm}а\hspace{0.04cm}з\hspace{0.04cm}а\hspace{0.04cm}н\hspace{0.04cm}и\hspace{0.04cm}е.\hspace{0.2cm} Пусть $\alpha=\angle ACK$.
\\Тогда \hspace{0.2cm} $\angle AKO = \frac{\pi}{2}-2\alpha$, и по теореме сину\vspace{0.7em}\\сов 
$\frac{AK}{\sin{\alpha}} = \frac{AC}{\sin{(\frac{\pi}{2}-2\alpha)}}$, \hspace{0.4cm} откуда \hspace{0.4cm} $10\sin^2{\alpha} +$\vspace{0.7em}\\ \noindent$ 23\sin{\alpha}-5 = 0$, т. е. $\sin{\alpha} = \frac{1}{5}$. Кроме того, $AK = AQ\tan{2\alpha} = BO\tan{2\alpha} = BM \tan^2{2\alpha}$
\\\textbf{4}. $a = 8$, $b = 56$, $c = 392$. У\hspace{0.04cm}к\hspace{0.04cm}а\hspace{0.04cm}з\hspace{0.04cm}а\hspace{0.04cm}н\hspace{0.04cm}и\hspace{0.04cm}е. \hspace{0.2cm} По \hspace{0.2cm} условию $b=aq$, $c=aq^2$, где $a$ и $q$ -- натуральные числа. \hspace{0.2cm} Из \hspace{0.2cm} делимости \hspace{0.2cm} чисел \hspace{0.2cm}$2240=2^6\cdot5\cdot7$ и $4312=2^3\cdot7^2\cdot11$ на $b$ и $c$ следует, что $q$ может приниматься одно из трех значений 2, 7 или 14.
\\\textbf{5}. $7/4$, $1/4$. У\hspace{0.04cm}к\hspace{0.04cm}а\hspace{0.04cm}з\hspace{0.04cm}а\hspace{0.04cm}н\hspace{0.04cm}и\hspace{0.04cm}е. Пользуясь соотношением $A\sin{\alpha} + B\cos{\beta} = \sqrt{A^2+B^2}\cdot\sin{\alpha+\upvarphi}$,убедитесь в том, что левая часть уравнения не больше $\sqrt{2}$ при любых $x$, причем она равна $\sqrt{2}$ только при $\ctg{2\pi x} = 1$.
\\\textbf{6}. $a = -1$; $a = 2$. У\hspace{0.04cm}к\hspace{0.04cm}а\hspace{0.04cm}з\hspace{0.04cm}а\hspace{0.04cm}н\hspace{0.04cm}и\hspace{0.04cm}е. Поскольку $3-2\sqrt{2} = \frac{1}{3+2\sqrt{2}}$ вместе с решением $(x_0;y_0)$ системе удовле\noindentтворяет также решение $(x_0;-y_0)$. Если решение единственно, то $y_0 = 0$.\vspace{1.7em}\\
\noindent Вариант 10
\\\textbf{1}. $\frac{\pi}{3}(6x\pm1)$,$k \in Z$.
\\\textbf{2}. $(2; (7+\sqrt{17})/4)$. \textbf{3}. 2:5. \textbf{4}.
 2,1 кг.
\\\textbf{5}. 8.  У\hspace{0.04cm}к\hspace{0.04cm}а\hspace{0.04cm}з\hspace{0.04cm}а\hspace{0.04cm}н\hspace{0.04cm}и\hspace{0.04cm}е. Высота $h$ пирамиды находится из равенства $V \frac{1}{3}rS$, где $V$ -- объем пирамиды, $S$ -- полная поверхность пирамиды, а $r$ -- радиус вписанного шара.\vspace{1.7em}\\
\noindent Вариант 11
\\\textbf{1}. $\frac{\pi}{12}(12k\pm5)$,$k\in Z$.
\\\textbf{2}. $(0; \log_{2}{3})$.
\\\textbf{3}. $[-1;2]\cup[3;4]$.
\\\textbf{4}. 90 $\sqrt{3}$.  У\hspace{0.04cm}к\hspace{0.04cm}а\hspace{0.04cm}з\hspace{0.04cm}а\hspace{0.04cm}н\hspace{0.04cm}и\hspace{0.04cm}е. Пусть $AO = x$, $DO = y$(рис. 9), поскольку $\frac{BC}{AD} = \frac{OC}{AO} = \frac{OB}{DO} = \frac{1}{2}$,
\vspace{1em}
\begin{figure}[H]
\captionsetup{singlelinecheck=off, margin=0.5cm, labelformat=empty}
\centering
        \begin{tikzpicture}[scale=0.6]
            \draw [](4.5,9.25) to[short] (12,9.25);
            \draw (4.5,9.25) to[short] (12,9.25);
            \draw [short] (4.5,9.25) -- (6.75,13.75);
            \draw [short] (12,9.25) -- (11.25,13.75);
            \draw [short] (6.75,13.75) -- (11.25,13.75);
            \draw [short] (6.75,13.75) -- (12,9.25);
            \draw [short] (11.25,13.75) -- (4.5,9.25);
            \draw [short] (8.75,12) -- (8.75,9.25);
            \draw [short] (11.25,13.75) -- (11.25,9.25);
            \node [font=\large] at (4.5,9) {A};
            \node [font=\large] at (6.5,14.25) {B};
            \node [font=\large] at (11.5,14.25) {C};
            \node [font=\large] at (12.25,9) {D};
            \node [font=\large] at (11.25,9) {E};
            \node [font=\large] at (8.75,8.75) {F};
            \node [font=\large] at (8.75,12.75) {O};
            \node [font=\normalsize] at (7,11.25) {x};
            \node [font=\normalsize] at (10,11.25) {y};
        \end{tikzpicture}
        \caption{\textit{Рис. 9.}}
\end{figure}
\noindentполучим $BC=8$, $OC=\frac{1}{2}x$, $OB=\frac{1}{2}y$. Далее,\vspace{1em}
$AC+BD=\frac{3}{2}(x+y)=36$. Кроме того, из треугольни\vspace{1em}ка $DEO$, где $EO \perp DE$, имеем $OE^2 = $
$ = OD^2 - DE^2$. Но $OE =\frac{\sqrt{3}}{2}x$, $DE=16-AE$ =
= $16-\frac{1}{2}x$. Поэтому $\frac{3}{4}x^2 = y^2-(16-\frac{1}{2}x)^2$.
\\\noindentОстальное ясно.
\\\textbf{5}. $(0; \frac{1}{54}$). У\hspace{0.04cm}к\hspace{0.04cm}а\hspace{0.04cm}з\hspace{0.04cm}а\hspace{0.04cm}н\hspace{0.04cm}и\hspace{0.04cm}е. Поскольку функция $a = f(x) = 2x^3+x^2-x-1$ \hspace{0.2cm} возрастает, \hspace{0.2cm} необходимо найти решения системы\vspace{0.7em}
\\$\begin{cases}
    12x^3-7x>6f(x)+1,\\
    x>0;
\end{cases}$
\vspace{0.7em}
\\\noindentа затем множество положительных значений $a$ при этих значениях $x$.
\vspace{4em}    
\\\textbf{Ф\hspace{0.04cm}и\hspace{0.04cm}з\hspace{0.04cm}и\hspace{0.04cm}к\hspace{0.04cm}а}
\\\textbf{Физический факультет}
\\\textbf{1}. $h=gt_{1}t_{2}(4t+t_{1}+t_{2})/(2(t_{1}+t_{2}))$.
\\\textbf{2}. $\omega_{\text{min}} \leq \omega \leq \omega_{\text{max}}$, где
\\$\omega_{\text{min}} = \sqrt{\frac{g}{R}\frac{\sin{\alpha}-\mu\cos{\alpha}}{(\cos{\alpha}+\mu\sin{\alpha})\sin{\alpha})}}$,\vspace{0.5em}
\\\noindent$\cos{\alpha} = \frac{R-h}{R}$,\vspace{0.5em}
\\$\omega_{\text{max}} = \sqrt{\frac{g}{R}\frac{\sin{\alpha}+\mu\cos{\alpha}}{(\cos{\alpha}-\mu\sin{\alpha})\sin{\alpha}}}$,\vspace{0.5em}
\\$\sin{\alpha}=\frac{\sqrt{(2R-h)/h}}{R}$\vspace{0.5em}
\\\noindent($\alpha$ - угол между радиусом, соединяющим шайбу с центром сферы, и вертикалью).
\\\textbf{3}. $V_{\text{погр}} / V_{\text{доски}} = 5/8$.
\\\textbf{4}. $\delta x_{\text{max}} = \frac{mg}{k}+\sqrt{(\frac{mg}{k})^{2}+\frac{2m^{2}gH}{k(M+m)}}$,\vspace{1em}
\\\noindentпричем колебания происходят около нового положения равновесия с координатой $x_{0}=(M+m)g/k$.
\vspace{1em}
\\\textbf{5}. $H =\frac{p_{0}}{\rho_{0}g}(\frac{L}{2d}-1)+\frac{L(\rho-\rho_{0})}{2\rho_{0}}-d$.
\vspace{0.9em}
\\\textbf{6}. $A =\nu R(T_{3}-T_{4})(T_{2}T_{4}-2T_{3}T_{4}+T_{3}T_{1})/(2T_{3}T_{4})$.
\\\textbf{7}. $I=\varepsilon I_{0}(\varepsilon+RI_{0}) = 0,1$ A.
\\\textbf{8}. $z_{n} = 2 \pi^2 Emn^2/(eB^2) > 0$, $n = 1, 2, 3, ...$



\egroup
\end{multicols*}
\vspace{2em}
\textbf{Случайная таблица:}

\begin{tabular}{|c|c|c|c|}
\hline
Формула 1 & Формула 2 & Формула 3 & Формула 4 \\
\hline
$c = \sqrt{a^2+b^2}$ & $\sin^2{x}+\cos^2{x} = 1$ & $\lim_{x\to\infty} \frac{1}{x}=0$ & $ax^2+bx+c=0$ \\\hline
$\lim_{x\to 0} \frac{\sin(x)}{x} = 1$ & $(a+b)^3=a^3+3a^2b+3ab^2+b^3$ & $\int_{-\infty}^{\infty}{e^{-x^{2}}dx} = \sqrt{\pi}$ & $S = v_{0}t+\frac{at^{2}}{2}$ \\\hline
\end{tabular}


